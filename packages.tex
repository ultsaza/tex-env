\usepackage{bm} 
\usepackage{amsmath} 
\usepackage{amssymb}
\usepackage{amsfonts}
\usepackage{latexsym}
\usepackage{mathtools}
\usepackage[dvipdfmx]{graphicx,color} 
\usepackage[dvipdfmx, % 自動的にハイパーリンクを作成するための設定
             bookmarks=true,
             bookmarksnumbered=true,
             bookmarkstype=true]{hyperref}
\usepackage{pxjahyper} % pLaTeXでhyperrefにおけるPDFの目次の文字化け問題を調整
\usepackage{url}
\usepackage{ulem}
\usepackage{subfigure}
\usepackage{comment} %\begin{comment} こんな感じでコメントアウト \end{comment}
\usepackage{physics}
\usepackage{listings}
\usepackage{booktabs}
\usepackage{array}
\usepackage{ascmac}
\usepackage{fancybox}
\usepackage{titlesec} % 見出しのフォントを変更するためのパッケージ
\usepackage{moreverb}
\usepackage{tikz}
\usepackage{here}
\usepackage{float}
\usepackage{tocbibind}
\usepackage{placeins}



% 見出しのフォントとスタイルの設定
\titleformat{\section}
  {\normalfont\Large\bfseries}{\thesection}{1em}{}
\titleformat{\subsection}
  {\normalfont\large\bfseries}{\thesubsection}{1em}{}
\titleformat{\subsubsection}
  {\normalfont\normalsize\bfseries}{\thesubsubsection}{1em}{}
  \titleformat{\paragraph}[runin] % runin は見出しを本文と同じ行に表示するオプション
  {\normalfont\bfseries} % 太字で通常のフォントスタイル
  {} % 見出し番号は表示しない
  {0em} % 見出しと本文の間のスペース
  {} % 追加のスタイルなし


% listingsパッケージの設定
\lstset{
    language=C,
    basicstyle=\ttfamily\small,
    keywordstyle=\color{blue}\bfseries,
    commentstyle=\color{gray},
    stringstyle=\color{red},
    tabsize=4,
    showspaces=false,
    showstringspaces=false,
    breaklines=true,
    numbers=left,
    numberstyle=\tiny\color{gray},
    stepnumber=1,
    numbersep=5pt
}
